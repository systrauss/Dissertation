\begin{table}[!]
    \centering
    \caption{Dynamic Moments of Inertia of Bands seen in $^{156}$Gd}
    \begin{tabular}{c|c}
        \toprule
        Band & Moment of Inertia  \\
        \hline
        Ground State & 0.0220 (11) \\
        1st excited $0^+$ & 0.0245 (13) \\
        2nd excited $0^+$ & 0.0187 (8) \\
        3rd excited $0^+$ & 0.0301 \\
        $\gamma$-band & 0.0259 (13) \\
        2nd excited $2^+$ & 0.0208 \\
        1st excited $4^+$ & 0.0242 (10)  \\
        \bottomrule
    \end{tabular}
    \footnotesize
    \item Table \ref{tab:156_Dynamic}: List of the moments of inertia of the bands seen in $^{156}$Gd in this experiment. The moment of inertia is the slope in a least-squares linear regression. Those without error only had one or two points of energy difference, so the standard deviation of the slope could not be calculated. The ground state band and the first excited $0^+$ band agree within two standard deviations. The same is true of the ground state band and both the $\gamma$-band and first excited $4^+$ band.
    \label{tab:156_Dynamic}
\end{table}