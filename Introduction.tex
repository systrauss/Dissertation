%
% Modified by Megan Patnott
% Last Change: Jan 18, 2013
%
%%%%%%%%%%%%%%%%%%%%%%%%%%%%%%%%%%%%%%%%%%%%%%%%%%%%%%%%%%%%%%%%%%%%%%%%
%
% Modified by Sameer Vijay
% Last Change: Tue Jul 26 2005 13:00 CEST
%
%%%%%%%%%%%%%%%%%%%%%%%%%%%%%%%%%%%%%%%%%%%%%%%%%%%%%%%%%%%%%%%%%%%%%%%%
%
% Sample Notre Dame Thesis/Dissertation
% Using Donald Peterson's ndthesis classfile
%
% Written by Jeff Squyres and Don Peterson
%
% Provided by the Information Technology Committee of
%   the Graduate Student Union
%   http://www.gsu.nd.edu/
%
% Nothing in this document is serious except the format.  :-)
%
% If you have any suggestions, comments, questions, please send e-mail
% to: ndthesis@gsu.nd.edu
%
%%%%%%%%%%%%%%%%%%%%%%%%%%%%%%%%%%%%%%%%%%%%%%%%%%%%%%%%%%%%%%%%%%%%%%%%


%
% Chapter 1
%

\chapter{Introduction}

Really, I have no idea what to write here. I suppose I should be writing vaguely about shape coexistence and then explaining key features of it. 

Nuclear structure is broad field, ultimately encompassing the entire nuclear chart, and contributing to other nuclear subfields. Of great interest are phenomena that appear in isolated areas of nuclei, instead of across the chart as a whole. Similarly, phenomena that appear in different sections of nuclei for differing reasons.

Low-lying $0^+$ states are one such phenomenon. In some nuclei, they are caused by particle-hole excitations, in others by intruder states, and in others still, by shape coexistence \citep{wood99:_e0}.

\section{Collectivity/Shape Coexistence}

% and here is the Wood et al 2011 paper.

\section{Electromagnetic Radiation}

\subsection{Internal Conversion}

% gimme dat proof

\subsubsection{E0 Transitions}

% gimme dat proof

\subsection{E0 Transitions for Nuclear Structure}

Electric monopoles are important in studying the nature of nuclear structure. Depending on the strength of the transition, E0 transitions can indicate different types of nuclear structure, ranging from shell model particle-hole pairings to shape coexistence. 

%need to start summarizing that Wood et al 1999 paper.

% % uncomment the following lines,
% if using chapter-wise bibliography
%
% \bibliographystyle{ndnatbib}
% \bibliography{example}