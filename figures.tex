%
% Modified by Megan Patnott
% Last Change: Jan 18, 2013
%
%%%%%%%%%%%%%%%%%%%%%%%%%%%%%%%%%%%%%%%%%%%%%%%%%%%%%%%%%%%%%%%%%%%%%%%%
%
% Modified by Sameer Vijay
% Last Change: Tue Jul 26 2005 13:00 CEST
%
%%%%%%%%%%%%%%%%%%%%%%%%%%%%%%%%%%%%%%%%%%%%%%%%%%%%%%%%%%%%%%%%%%%%%%%%
%
% Sample Notre Dame Thesis/Dissertation
% Using Donald Peterson's ndthesis classfile
%
% Written by Jeff Squyres and Don Peterson
%
% Provided by the Information Technology Committee of
%   the Graduate Student Union
%   http://www.gsu.nd.edu/
%
% Nothing in this document is serious except the format.  :-)
%
% If you have any suggestions, comments, questions, please send e-mail
% to: ndthesis@gsu.nd.edu
%
%%%%%%%%%%%%%%%%%%%%%%%%%%%%%%%%%%%%%%%%%%%%%%%%%%%%%%%%%%%%%%%%%%%%%%%%


%
% Chapter 1
%

\chapter{Introduction}

\section{Coincidence Figures}

\begin{figure}
    \centering
    \includegraphics[scale=0.3]{Figures/156Gd_2to0.png}
    \caption{Level Scheme of $^{156}$Gd. The gamma ray of the $2^+$\rightarrow$0^+$ transition in the ground state was gated on. It was then compared with the gated spectrum from the gamma ray of the $4^+$\rightarrow$2^+$ transition in the ground state. Peaks only appearing in the first gate were assumed to go into the $2^+$ state, and assignments were made. Due to the low energy of the $2^+$\rightarrow$0^+$ transition, the efficiency was lower, and it is likely that transitions into the $2^+$ state were missed. The levels are organized by band. The lower levels of the band, unseen by gamma rays in this gate, are in gray.}
    \label{fig:156_2to0}
\end{figure}

\begin{figure}
    \centering
    \includegraphics[scale=0.28]{Figures/156Gd_4to2.png}
    \caption{Level Scheme of $^{156}$Gd. The gamma ray of the $4^+$\rightarrow$2^+$ transition in the ground state was gated on. It was then compared with the gated spectrum from the gamma ray of the $6^+$\rightarrow$4^+$ transition in the ground state. Peaks only appearing in the first gate were assumed to go into the $4^+$ state, and assignments were made. Additionally, these peaks were also gated on, to look for cascades leading into the $4^+$ state, which were found in several cases. The levels are organized by band. The lower levels of the band, unseen by gamma rays in this gate, are in gray.}
    \label{fig:156_4to2}
\end{figure}

\begin{figure}
    \centering
    \includegraphics[scale=0.28]{Figures/156Gd_6to4.png}
    \caption{Level Scheme of $^{156}$Gd. The gamma ray of the $6^+$\rightarrow$4^+$ transition in the ground state was gated on. It was then compared with the gated spectrum from the gamma ray of the $8^+$\rightarrow$6^+$ transition in the ground state. Peaks only appearing in the first gate were assumed to go into the $6^+$ state, and assignments were made. Additionally, these peaks were also gated on, to look for cascades leading into the $6^+$ state, which were found in several cases. The levels are organized by band. The lower levels of the band, unseen by gamma rays in this gate, are in gray.}
    \label{fig:156_6to4}
\end{figure}

\begin{figure}
    \centering
    \includegraphics[scale=0.3]{Figures/156Gd_8to6.png}
    \caption{Level Scheme of $^{156}$Gd. The gamma ray of the $8^+$\rightarrow$6^+$ transition in the ground state was gated on. It was then compared with the gated spectrum from the gamma ray of the $10^+$\rightarrow$8^+$ transition in the ground state. Peaks only appearing in the first gate were assumed to go into the $8^+$ state, and assignments were made. Additionally, these peaks were also gated on, to look for cascades leading into the $8^+$ state, which were found in several cases. The levels are organized by band. The lower levels of the band, unseen by gamma rays in this gate, are in gray.}
    \label{fig:156_8to6}
\end{figure}